% Initial cut at documentation for GRCAMAIN, a program for
% the Casio fx-9750G-class graphing calculators.
%
\def\trade{\textsuperscript{\textregistered}}
\author{brickviking}
\title{Help for Grocery program}
\date{2018-10-11}
\documentclass[]{article}

\usepackage{siunitx}
% I need to look up docs for sisetup
% Seems I can't get it to do 15,495.00 (doesn't do commas in numbers)
% TODO: find alternative for siunitx
\sisetup{
    round-mode          = places, % Rounds numbers
    round-precision     = 2, % to 2 places
}


\usepackage{multirow}
\begin{document}
\maketitle
	
% Intro
\section{Introduction}

GROCERY, hereafter referred to as simply GRC, is a small collection of Casio BASIC programs
to help manage the grocery totals each time you go shopping. The collection includes {\bf GRCAMAIN} 
(the main program), {\bf GRCBTN} (initial screen creation), {\bf GRCHELP} (the help screens), {\bf GRCLIST},
and a small selection of other helper programs to go along for the ride. The programs I've listed so far
can be run on their own, the other programs should probably not be run straight from the Programs listing.
{\bf GRCXBUGS } is not actually runnable code, but is a listing of bugs that I know about so far. GRCHELP can be
left off if your calculator doesn't have much room left, but is useful while you get to know GRC.


% Installation
\section{Installation}

To install GRC, you'll need a Casio {\bf calculator} (fx-9750G/9860G or later), the {\bf cable} to connect
it to a computer, the {\bf FA-124 program} (a free download from Casio), {\bf Microsoft\trade Windows\trade}, and
some {\bf batteries} for your calculator. If you're missing any of these things, GRC may not work well for you.
% Insert pretty pictures of the bits we'll need.

First, make sure your computer's running, and that you've installed the FA-124 program.
The calculator comes with a cable (serial or USB2) that you can { \it usually \/} plug
into your computer, if your computer has the right kind of connector. Plug one end into
the calculator and plug the other end into the computer. If your cable is USB, your calculator
should then prompt you to select the Connection mode, choose {\bf F1} (DataTrans) here.

Find and start up the FA-124 program, and look for the "Connect" button. Click this, and
your calculator should connect.
% Need pretty picture of FA-124 with connected calculator
% Need better noob-friendly instructions for importing GRCAMAIN.g1m
All the GRC programs have filenames beginning with GRC, so you'll need to copy all of these to the calculator with FA-124. The Casio manual
describes how to copy programs, so if you get stuck, check the manual. Don't forget to select Disconnect afterwards. Once again, if you're low
on room, you can leave off GRCHELP, but don't forget to edit GRCAMAIN in this case, as one of the program menu entries points to that program.


% Main
\section{Starting}
\subsection{Initial startup}
Turn on calculator, hit the {\bf MENU } key, select {\b B } (or Programs). You'll see a list of
installed programs. Skip down to GRCBTN, hit \b F1. This creates the initial framework of the screen,
and you only have to do this once.

\subsection{Starting regularly}
After that, every time you want to run the program, select GRCAMAIN, and hit F1 (EXE).
This loads up the main program and shows you the first screen, asking if you wish to
load previously-saved values. If this is your first time, you won't have any, so just
select No in this case. 
% Requests whether to load previously-saved values (Yes/No)


\section{Main Program}
% Screen
\subsection{Main screen}
Once you've started the program, you'll see the main screen. The left hand side
shows the last entry made, the five categories of groceries, their respective totals
in dollars and cents, and a counter for each category. Underneath that, you'll see the
Grand Total, and the total item count. This Grand Total may or may not include the tax
figure, depending upon your Tax settings in the program.

On the right hand side, you'll see a short list of temporary values---more on that later,
and the currently-applied tax rate, along with the amount of tax calculated. On the lower part of
the screen, you'll see the overall total cost and the overall count of items, and below that, the menu line.
% Pretty picture of whole screen, with highlights of sections
% TODO: provide initial screen, and while-running screen
\subsection{Menu bar}
% Need pretty picture of bottom of screen plus top line of keys including F1-F6
The Menu bar is where most of the action happens, and works very much like the menus
in the calculator; the only differences being the words and lack of decoration.
The {\bf F1} through {\bf F5} keys are used to add entries to categories % yeesh!!
and the  {\bf F6} key runs the Help program. For the Vege and Meat categories,
you'll get asked if you're putting in a count (how many there are) or a weight (how
much does your item weigh). For all the categories, you'll also get asked for the
unit cost (per KG if you supplied weight in the previous step).
% Need pretty picture here describing GRCITEM/GRCWGT usage
\subsection{Help}
This is a complete though concise explanation of program functions. The usual menu
is on the bottom, and the help program has a complete list of contents (over two pages),
followed by individual help pages for each major function. The one wrinkle with the
Help program contents is that you need to choose a section by number, not by Function key.
To leave the Help, simply choose {\bf Exit} (F1) from the menu. To page forward through
the help, use {\bf Next} (F4), to page backwards, use {\bf Back} (F3), and to skip back
to the Contents, hit F6 on any help page.
% Functions
\section[Functions]{Functions Menu}
% Prefs
\subsection[Preferences]{Preferences Menu}

% Tax
\subsubsection[Tax]{Tax Menu}
As such, this program calculates one tax rate for everything you enter in; it takes no
account of items that are taxed differently or perhaps not at all. Additionally, there
are two ways to show the tax, either the grand total includes the tax as it would do in
the European Union, Great Britain, Australia, New Zealand and similar countries, or it
is calculated separately and {\it added} to the category total. For example in New Zealand,
the tax rate is 15\% of the total price, and the ticket price at the time of purchase is
the only amount you will pay. In America the tax is calculated additionally to the shelf
price, and consumers are required to pay some additional forms of tax depending upon where
they live. This can mean a significant difference between the shelf price of an item and
the till price. Table 1 shows an example of three common items and their respective
totals at the till.

\begin{table}
  \begin{center}
	\label{tab:table1}
	\begin{tabular}{l|S|S|S|S} % <-- Alignments: 1st column left, 2nd and onwards middle, with vertical lines in between
		{\textbf{Item}} & \multicolumn{2}{c}{ \textbf{Ticket Price}} & \multicolumn{2}{c}{ \textbf{Total}}\\
		 & {(NZD) } & {(USD)} & {(NZD)} & {(USD)} \\
		\hline
		Apples & 3.95 & 1.75 & 3.95 & 2.38 \\
		Shoes & 129.99 & 59.99 & 129.99 & 65.42 \\
		Car & 11995.00 & 5995.00 & 11995.00 & 7377.52 \\
	\end{tabular}
	\caption{Comparing shelf and till prices.}
\end{center}
\end{table}

\section{Conclusions}
To my knowledge, this is the only program of its type listed at casiopeia.net,
and is a good chance to exercise the power of this calculator. At least in this
country, a lot of school attendees get told to buy them so they can complete
mathematical studies; after they leave school they never really seem to use them
again. I've definitely got a reason to give these calculators a second life.

\end{document}
